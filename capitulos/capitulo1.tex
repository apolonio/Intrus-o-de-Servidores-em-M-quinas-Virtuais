
\chapter{Introdução}
Existem diversas ameaças que atingem a Web nos dias atuais e aumentam cada vez mais com o passar do tempo. Com a necessidade de criação de novas funcionalidades em um ambiente em crescimento, essa problemática permiti a invasão de diversos serviços hospedados em servidores. O presente trabalho visa demonstrar algumas técnicas utilizadas para invadir serviços hospedados em máquinas virtuais\cite{laureano}. Assim é necessário, conhecimento sobre conceitos de redes, Linux, máquinas virtuais e ferramentas de intrusão. Para isso exploraremos o Backtrack uma ferramenta baseada no WHAX, Whoppix ~\cite{giavaroto} e o Kali~\cite{broad} Linux utilizada para testes de penetração muito utilizada por auditores, analistas de segurança de redes e sistemas, hackers éticos, etc.
No que tange as máquinas virtuais procura-se discutir as vantagens e desvantagens considerando as diversas configurações de virtualização e para-virtualização.